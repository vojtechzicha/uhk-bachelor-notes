\epigraph{Vytvořeno podle přednášek \textbf{Ing. Martina Krále} v kombinovaném studiu v zimním semestru 2021/2022 a podle materiálů dostupných ke studiu v systému UHK Oliva.}{}

\section{Úvod do teorie mezinárodního obchodu}

\textbf{Zahraniční obchod} je úsek národního hospodářství, který zahrnuje směnu (hmotných a nehmotných statků) se zahraničím. \textbf{Mezinárodní obchod} je poté souhrn zahraničních obchodů jednotlivých států. Mezinárodní obchod má význam pro \textit{efektivnost}, \textit{proporcionalitu}, \textit{demonstrativní efekt}, podporu mírové spolupráce, snížení rizika konfliktu a růst vzdělanosti.

Platí, že pro větší státy je míra závislosti na vnějších ekonomických vztazích menší. K mezinárodnímu obchodu může být \textit{pasivní} nebo \textit{aktivní} přístup. Pro \textbf{aktivní pojetí politiky zahraničního objektu} je vývoz základní hnací sílou ekonomického rozvoje - stát projevuje úsilí o co nejsnazší přístup na trhy jednotlivých zemí, upravuje své zákony a dekrety, organizuje a podporuje výstavní a veletržní činnosti.

Na vnější ekonomické vztahy působí \textit{objektivní vlivy} (geografická poloha, klimatické faktory, vyspělost, vzdělanost, apod.) a \textit{subjektivní vlivy} (vliv státu a institucí, rozsah státních zásahů). Stát může aplikovat liberální nebo protekcionistickou politiku vůči zahraničnímu obchodu, definovanou jako \textit{státní regulaci}. Rozsah a síla státních zásahů roste s relativní ekonomickou slabostí země.

Na vnější ekonomické vztahy působí různé \textbf{subjekty} - centrální vláda, parlament, politické strany, bankovní instituce, silové složky, asociace podnikatelů a hospodářské komory, odborové svazy a velké průmyslové podniky.

\subsection{Kvantifikace zahraničního obchodu}

\begin{itemize}
    \item Užitý spotřebitelský produkt - \textit{vyrobený + dovoz - vývoz}
    \item Vyrobený spotřebitelský produkt - \textit{užitý - dovoz + vývoz}
    \item Objem zahraničního obchodu - \textit{v měnové jednotce}
    \item Schopnost vývozu pokrýt dovozní potřeby - \textit{vývoz / dovoz}
    \item Vztah dovozu vůči devizovým rezervám - \textit{mezinárodní likvidita / dovoz}
    \item Poměr vývozu k dluhové službě
    \item Míra otevřenosti ekonomiky - \textit{podíl zahraničního obchodu vůči HDP}
\end{itemize}

\subsection{Státní zásahy}

Stát může zasahovat do mezinárodního obchodu různými možnostmi.

\begin{itemize}
    \item Vydání \textbf{dlouhodobých plánů} na 20 let, \textbf{střednědobých plánů} na programové období (cca 5 - 7 let) a \textbf{ročních plánů} pro stanovení podporovaných složek obchodu.
    \item Vynucováním \textbf{daňové politiky} s původně fiskální, nově spíše regulační funkcí. Vláda nabízí daňové úlevy a prázdniny.
    \item Prováděním \textbf{měnových finančních nástrojů} a nepřímo dopadem stavu veřejných financí. Úroková sazba centrální banky má vliv na zahraniční obchod - růst sazby vyhovuje dovozním obchodním společnostem.
    \item Stavem \textbf{zahraničněobchodní politiky} - nastavením obchodní politiky, podpisy mezinárodních dohod, diskriminace a preference vybraných partnerů. Stát má k dispozici \textit{aktivní prostředky} (vývozní prémie, úvěrování vývozu, úřady v zahraničí, agentury) a \textit{pasivní prostředky - překážky} (cla, kvantitativní restrikce, licenční řízení).
    \item Vynucováním \textbf{administrativně-právních nástrojů}, jejichž vliv výrazně stoupl - etikety, návody, bezpečnost práce a ochrana zdraví, volný pohyb osob a vízová politika.
\end{itemize}

\section{Historie teorií mezinárodního obchodu}

V průběhu historie existovalo několik teorií mezinárodního obchodu.

\subsection{Merkantilismus}
\epigraph{16. - 18. století}{}

V souvislosti s koloniální expanzí evropských mocností se ustavila teorie merkantilismu. Nehledá objektivní pravdu, ale způsoby, jak zlepšit postavení státu v konkurenci ostatních zemí. Bohatství národa vyjadřují zásobou drahých kovů a považují aktivní obchodní bilanci jako cestu k čistému přílivu drahých kovů do domácí ekonomiky.

V teorii existují rozpory - peníze ve formě rezerv nezvyšují bohatství národa, tudíž čistý příliv drahých kovů vede k poklesu úrokové sazby a růstu cenové hladiny.

Radikální křídlo merkantilistů (tzv. \textit{bullionisté}) tlačí na devizové kontroly a zákaz vývozu drahých kovů. Primární proud podporuje vývoz výrobků s přidanou hodnotou a dovoz surovin a potravin.

\begin{itemize}
    \item \textbf{Thomas Munn} - představitel primárního proudu.
\end{itemize}

\subsection{Klasická politická ekonomie}
\epigraph{18. století}{}

Podle reálných dopadů merkantilismu se definuje klasická politická ekonomie. Kritický proud obsahoval myšlenky \textbf{fyziokratů} a \textbf{utilitaristů}.

\begin{itemize}
    \item \textbf{David Hume} - formulace \textit{kvantitativní teorie peněz}, zahraniční obchod může být výhodný pro obě země a země, která se brání dovozu, poškozuje především sama sebe.
    \item \textbf{Adam Smith} - zakladatel ekonomie, filozofie liberalismu a neviditelné ruky trhu; reagoval na manufaktury a dělbu práce.
\end{itemize}

Adam Smith definoval \textbf{princip absolutní výhody} - země by se měla specializovat na výrobu těch výrobků, které je schopna vyrábět levněji než ostatní země. Předpokládá svobodný obchod mezi zeměmi, nulové transakční náklady, pracovní teorii hodnoty (vše lze vyjádřit lidskou prací), konstantní výnosy z rozsahu, dokonalou mobilitu pracovní síly mezi odvětvími, dokonalou imobilitu pracovní síly mezi zeměmi a indiferentní produkt.

\begin{example}\begin{displayquote}
Domácí ekonomika vyrábí chléb za 20 člověkohodin, pivo za 40 člověkohodin. Zahraniční ekonomika chléb za 10 člověkohodin, pivo za 60 člověkohodin. Jaké je optimální rozdělení ekonomiky? Domácí ekonomika má vyrábět pivo, zahraničí chléb.

Výsledné ceny však budou uvnitř intervalů (např. chleba 1/3 piva, pivo 3 chleby), dojde k mezinárodní směně. Poté můžeme zjistit výhody specializace - domácí ekonomika vyrobí místo 2 chlebů 1 pivo a v zahraničí jej vymění za 3 chleby (+ 50 \%); zahraniční ekonomika vyrobí místo 1 piva 6 chlebů, které vymění za 2 piva (+ 100 \%).
\end{displayquote}\end{example}

\begin{example}\begin{displayquote}
Domácí ekonomika vyrábí chléb za 10 člověkohodin, pivo za 20 člověkohodin. Zahraniční ekonomika chléb za 40 člověkohodin, pivo za 30 člověkohodin. Jaké je optimální rozdělení ekonomiky? Domácí ekonomika má vyrábět chléb, zahraničí pivo. Ačkoli je zahraniční ekonomika dražší, mezinárodní dělba práce se vyplatí.

Výsledné ceny však budou uvnitř intervalů (např. chleba 1 pivo, pivo 1 chléb), dojde k mezinárodní směně. Poté můžeme zjistit výhody specializace - domácí ekonomika vyrobí místo 1 piva 2 chleby a v zahraničí jej vymění za 2 piva (+ 100 \%); zahraniční ekonomika vyrobí místo 3 chlebů 4 piva, které vymění za 4 chleby (+ 33 \%).
\end{displayquote}\end{example}

\subsection{Základní principy klasické ekonomie}

\textbf{Heckscherův a Ohlinův model} přidává do teorie komparativních výhod kapitál. Předpoklady nově zahrnují odlišnost vybavenosti zemí výrobním faktory, rozdělení výrobků dle náročnosti na kapitálově nebo pracovně náročné. Dále předpokládáme, že výrobní technologie jsou pevně dány pro všechny země, nelze tedy při výrobě nahrazovat práci kapitálem a naopak, a pohyblivost výrobních faktorů mezi zeměmi je značně omezená.

\begin{itemize}
    \item Země, která je relativně hojněji vybavena kapitálem, bude mít kapitál relativně levnější. Tato země by se měla zaměřit na kapitálově náročné výrobky.
\end{itemize}

\textbf{Stolperův a Samuelsonův teorém o změně světových cen} definuje vztahy kapitálu a práce. Růst světové ceny kapitálově náročné komodity vyvolá růst ceny kapitálu. Tento růst vede ke snaze zvýšit výrobu této komodity na úkor pracovní komodity - toto vede k rozšíření výroby kapitálově výroby, zvýšení vývozu, zlepšení směnných relací a k růst dovozu. Růst ceny kapitálové výroby povede ke změně i u zemí zaměřených na pracovní výrobu, zhoršení směnných relací a vyvolání cyklového efektu.

\textbf{Rybczynského efekt} popisuje změnu relativní vybavenosti společnosti. Zvýšení poměru kapitálu a práce povede k relativnímu poklesu ceny kapitálu a relativnímu zvýšení ceny práce, což vyvolá změnu struktury domácí výroby ve prospěch kapitálově náročných výrobků.

\textbf{Leontiefův paradox} ověřoval platnost \textit{Heckscherova a Ohlinova modelu} na ekonomice USA. Oproti očekávání ukázal, že země (v roce 1947) vyváží spíše pracovně náročné výrobky a dováží kapitálově náročné.

\textbf{Teorie dětských odvětví Fridricha Lista} definuje, že země by se měla plně otevřít působení zahraniční konkurence až poté, co se její průmysl stane dospělým / konkurenceschopným. Problém teorie je s praktickou aplikací - dětská odvětví se sama brání vlastnímu dospívání.

\textbf{Teorie zbídačujícího růstu} definuje, že firmy v rozvojových zemích reagují na změnu světové ceny jejich produkce opačně, než odpovídá racionálnímu očekávání - při snížení ceny zvyšují objem výroby a vývozu, aby tak kompenzovali snížení příjmů v důsledku světové ceny.

\textbf{Teorie periferní ekonomiky} definuje, že ceny surovin a základních potravin s nízkou přidanou hodnotou rostou pomaleji než ceny průmyslových výrobků s vysokou přidanou hodnotou. Rychlejší růst cen vyspělých výrobků vede k otevírání nůžek mezi kapitálovými a periferními ekonomikami.

Teorie \textbf{rostoucích výnosů z rozsahu} vychází z klasického i neoklasického pojetí mezinárodního obchodu - v praxi předpokládáme rostoucí výnosy z rozsahu, což vede ke snížení jednotkových nákladů. Rostoucí výnosy přispívají ke zvýšení konkurenceschopnosti velkých firem.

\textbf{Teorie nové ekonomické geografie} definuje, že se firmy budou v národním, ale i mezinárodním měřítku geograficky sdružovat do \uv{pólů rozvoje}, které umožní využívat výhody - rostoucí výnosy z rozsahu, monopolistická konkurence, dopravní náklady.