\epigraph{Vytvořeno podle přednášek \textbf{JUDr. Bc. Jana Janečka, Ph.D.} v kombinovaném studiu v zimním semestru 2021/2022 a podle materiálů dostupných ke studiu v systému UHK Oliva.}{}

Právo 2 se orientuje na český právní řád a soukromoprávní oblasti - občanské, pracovní a procesní právo.

Rozeznáváme následující normy:

\begin{itemize}
\item \textbf{kogentní} - znamená domluvený normou, vztah mezi osobami nelze řešit dohodou, ale musí být uplatňováno podle textu normy; typicky norma veřejného práva;
\item \textbf{dispozitivní} - normy které mohou dohodou být upraveny; typicky norma soukromého práva.
\end{itemize}

Pro soukromé právo platí je dovoleno vše, co není zakázáno.

\section{Orgány ochrany práva}

Podstata \textbf{soudní činnosti} v oblasti soukromého práva znamená, že právní autorita nahradí svou jedinou vůlí, či svým jediným právním názorem dva právní názory sporných stran. Tím odstraní nejednotnost či rozpornost vůlí (spor), které mají být ve shodě (smlouva). \textbf{Civilní sou}d je druhem právní autority, která rozhoduje ve sporných a nesporných věcech v oblasti práva soukromého. 

\begin{itemize}
    \item \textbf{Civilní soudní proces} - 1. stupeň okresní soudy / krajské soudy pomocí \textit{žaloby}; 2. stupeň krajské soudy / vrchní soudy pomocí \textit{odvolání}; 3. stupeň Nejvyšší soud pomocí \textit{dovolání}.
    \item \textbf{Správní soudní proces} - následuje po správním řízení; 1. stupeň krajské soudy pomocí \textit{správní žaloby}; 2. stupeň Nejvyšší správní soud pomocí \textit{kasační stížnosti}.
    \item \textbf{Rozhodčí soudy} - Rozhodčí soud při Hospodářské komoře a Agrární komoře ČR pomocí \textit{rozhodnutí z rozhodčí smlouvy nebo doložky}.
\end{itemize}

\subsection{Občanské právo procesní}

V rámci civilního procesu pracujeme s následující pojmy:

\begin{itemize}
    \item \textbf{Procesní subjekty} jsou subjekty, které svými procesními úkony ovlivňují průběh civilního procesu. Základními jsou soud a účastníci, poté příp. zvláštní procesní subjekty.
    \item \textbf{Procesní úkony} jsou projevy vůle subjektů směřující k zahájení, průběhu a ukončení civilního procesu (např. žaloba, rozsudek, usnesení, předvolání, doručování apod.).
    \item \textbf{Procesní vztahy} jsou vztahy vznikající mezi procesními subjekty v průběhu civilního procesu.
\end{itemize}

Nárok stran na vydání rozhodnutí nemá svůj původ ve zvláštním procesním vztahu, ale vyplývá z úřední povinnosti soudního orgánu. Neuplatňuje se zde žádná vzájemná korelace subjektivních práv a povinností. Tato teorie popírá možnost soudu vydat konstitutivní rozhodnutí, soud rozhoduje pouze o \uv{opodstatněnosti} a \uv{neopodstatněnosti} nároku. Proces je jako boj stran – jedna se snaží využít slabosti druhé, samotná pravda může být popřena chytrým bojem. Podporuje se tedy především soupeření stran, soud rozhoduje pouze o opodstatněnosti nároku (\uv{chytřejší vyhrává}, kdo je \uv{procesně schopnější}).

\subsection{Sporná a nesporná řízení}

\textbf{Sporné řízení}  je občanským soudním řádem definováno jako soudní řízení o dvoustranných právních vztazích. Stranami tohoto řízení jsou \textit{žalobce} a \textit{žalovaný}. Sporné řízení je řízením \textit{kontradiktorním}, při němž se \uv{utkávají} dvě soupeřící strany, žalobce a žalovaný, a soud mezi těmito stranami rozhoduje jako nestranný \uv{rozhodčí}; jednání přitom sice řídí, ale je silně omezen vůlí stran.

Ve sporném řízení se uplatňují zásady řízení, které jsou pro řízení před soudem typické:

\begin{itemize}
    \item \textbf{Zásada dispositivní} znamená, že soud nemůže řízení zahájit z vlastní vůle ani z úřední povinnosti, ale pouze na návrh procesní strany – žalobce. Podobně je tomu u skončení řízení: rozhodne-li se žalobce ve sporu nepokračovat a žalobu vezme zpět, soud musí řízení zastavit.
    \item \textbf{Zásada projednací} znamená, že soud provádí jen ty důkazy, které navrhnou strany.
    \item \textbf{Zásada ústnosti} znamená, že jako podklad rozhodnutí lze použít jen to, co bylo při jednání u soudu ústně předneseno.
    \item \textbf{Zásada písemnosti} znamená, že se všechny procesní úkony stran i soudu protokolují.
    \item \textbf{Zásada přímosti} znamená, že soud může činit skutková zjištění jen z těch důkazů, které sám provedl.
    \item \textbf{Zásada rovnosti} stran garantuje všem účastníkům řízení v principu stejná procesní práva.
    \item \textbf{Zásada nestrannosti a nezávislosti soudu} znamená, že soudcem určité věci nesmí být ten, kdo má zvláštní poměr k účastníkům řízení, tzn. kdo je k nim ve vztahu příbuzenském, přátelském nebo nepřátelském, nebo kdo má zvláštní zájem na výsledku projednávaného sporu, tj. kdo je podjatý.
    \item \textbf{Zásada in dubio pro reo} (tzn. v pochybnostech ve prospěch žalovaného) znamená, že dospěje-li soud k závěru, že z provedeného dokazování nelze zjistit, je-li žaloba po právu, musí žalobu zamítnout.
    \item \textbf{Zásada formální pravdy} znamená, že soud nemusí zjišťovat přesně a úplně skutkový stav, ale spokojí se s takovým stupněm jeho poznání, který vyplyne z provedeného dokazování nebo na němž se shodnou strany řízení.
    \item \textbf{Zásada hospodárnosti} (zásada procesní ekonomie) znamená, že soud při řízení postupuje co nejhospodárněji tak, aby byly náklady řízení i zátěž pro strany minimální možné a přitom nebyl ještě zmařen účel řízení.
    \item \textbf{Zásada volného hodnocení důkazů} nestanoví žádná formální pravidla pro to, jak má soud důkazy hodnotit; soudce musí vycházet jen ze svého vnitřního přesvědčení.
    \item \textbf{Zásada arbitrárního pořadí důkazů} znamená, že soud provádí jednotlivé důkazy v libovolném pořadí, jak je to účelné (např. jak se k jednání dostavují jednotliví svědkové).
    \item \textbf{Zásada veřejnosti} stanoví, že sporné řízení (tak jako všechna řízení u soudu s výjimkou řízení dědického) je vždy veřejné.
    \item \textbf{Zásada právní pomoci} znamená, že nikdo nesmí utrpět újmu na svých právech proto, že by neznal v řízení před soudem svá procesní práva. Soudy jsou proto povinny ty účastníky řízení, kteří nejsou zastoupeni advokátem, o jejich procesních právech náležitě poučovat.
\end{itemize}

\textbf{Nesporné řízení} je druhou základní formou občanského soudního řízení. Jeho podstatným znakem je, že proti sobě nestojí žalobce a žalovaný, ale okruh účastníků závisí na druhu řízení a může být značně různorodý. Soud tu nemá roli pouhého rozhodce jako v řízení sporném, ale musí být sám aktivní a jednat tak, aby bylo dosaženo účelu řízení. Typickým příkladem nesporného řízení je řízení o svěření dítěte do péče jednoho z rodičů, řízení o soudní úschově nebo řízení ve věcech obchodního rejstříku.

V nesporném řízení se uplatňují zčásti odlišné zásady řízení, než jsou zásady typické pro řízení sporné:

\begin{itemize}
    \item \textbf{Zásada oficiality} \textit{(oproti zásadě dispositivní)} znamená, že soud jedná z úřední povinnosti. Řízení musí zahájit i bez návrhu, jestliže se hodnověrně dozví, že vznikla jeho potřeba.
    \item \textbf{Zásada vyhledávací} \textit{(oproti zásadě projednací)} znamená, že soud může provádět takové důkazy, které uzná za potřebné. Není při tom vázán návrhy účastníků.
    \item \textbf{Zásada materiální pravdy} \textit{(oproti zásadě formální pravdy)} znamená, že soud se musí snažit o přesné a úplné zjištění skutkového stavu. Nesmí se spokojit s tvrzeními účastníků, byť by i byla shodná.
\end{itemize}

Nesporné řízení je řízením inkvizičním, při němž soud hledá takové řešení projednávané věci, které bude v souladu s veřejným zájmem. Soudce tu není \uv{rozhodčím}, ale mocenským, vrchnostenským orgánem, jenž v řízení zastupuje stát.

\section{Rodinné právo}

\textbf{Rodinné právo} tvoří souhrn právních předpisů, jejichž předmětem jsou práva a povinnosti osob, které v různých rolích vystupují v rodině a ve společenstvích rodině obdobných; jde především o úpravu práv a povinností mezi manžely navzájem, mezi rodiči a dětmi a mezi dětmi a dalšími osobami, které jim nahrazují rodiče, či jejich funkci nějak doplňují. Rodinné právo se zabývá zejména manželstvím a registrovaným partnerstvím, manželským majetkovým právem, příbuzenstvím a švagrovstvím, jinými formami péče o děti, výživným.

\subsection{Manželství}

Manželství je definováno jako \textit{\uv{trvalý svazek muže a ženy vzniklý způsobem, který stanoví tento zákon}}. \textit{Účelem} manželství je založení rodiny, řádná výchova dětí a vzájemná podpora a pomoc.
 
\subsubsection{Sňatek}
 
\textbf{Sňatek} je jediný zákonem dovolený právní institut uzavření manželství. \textbf{Osobní náležitosti} sňatku jsou vymezeny negativně - nejsou dány překážky manželství. \textbf{Sňatečný projev vůle} je svobodný, úplný, souhlasný projev s nutností osobní přítomnosti snoubenců. \textbf{Sňatečný obřad }je slavnostní (obsahuje proslov oddávajícího, důstojnost chování přítomných) a veřejný (je učiněn na kterémkoli přístupném místě) a koná se v přítomnosti dvou svědků. \textbf{Svědek} sňatku je poté fyzická osoba mající způsobilost k právnímu jednání, zrak, sluch a znalost jazyka. Manželství poté vzniká ve chvíli kladných odpovědí (nebo jiného projevu vůle) obou snoubenců.

Lze provést \textbf{předsňatkové řízení}, které obsahuje nutnost vyplnit žádost na matrice, sdělit rozhodné skutečnosti a prokázat, že není přítomna žádná překážka. Úřad poté vydá osvědčení o statečné způsobilosti (důležité pro církevní sňatky, platnost 6 měsíců).

\begin{itemize}
    \item \textbf{Občanský sňatek} - uzavírá se před orgánem veřejné moci s nutnou přítomností matrikáře.
    \item \textbf{Církevní sňatek} - uzavírá se před orgánem církve / náboženské společnosti, která je k tomu oprávněna jiným právním předpisem; je zde nutnost předložit osvědčení vydané matričním úřadem (ne starší než 6 měsíců). Oddávající je poté povinen do 3 dnu od uzavření manželství doručit matričnímu úřadu, v jehož správním obvodu bylo manželství uzavřeno, protokol o uzavření manželství.
\end{itemize}
 
V případě přímého ohrožení života snoubence mohou oddávat velitelé námořních plavidel a letadel i cizí státní příslušníky a velitel vojenské jednotky České republiky v zahraničí tehdy, je-li jeden ze snoubenců českým občanem.

V případě zájmu \textbf{uzavřít manželství v cizině} je nutno předložit osvědčení o statečné způsobilosti. Sňatek je v ČR uznán po předložení osvědčení z ministerstva zahraničí země, kde byl sňatek uzavřen zvláštní matrice v Brně vedoucí evidenci manželství v cizině. Občan ČR muže uzavřít manželství také před diplomatickou misí nebo konzulárním úřadem ČR.

Sňatek má poté následující právní následky:

\begin{itemize}
    \item nemožnost vstoupit do dalšího manželství / registrovaného partnerství,
    \item aktivní legitimace k rozvodu,
    \item vznik společné jmění manželů,
    \item domněnka otcovství.
\end{itemize}
 
\subsubsection{Překážky vzniku sňatku}

Zákon definuje následující \textbf{absolutní zákonné překážky manželství}, které vylučují, aby osoba uzavřela manželství s kýmkoli:

\begin{itemize}
    \item \textbf{Nedostatek věku} - manželství nemůže uzavřít nezletilý, který není plně svéprávný. Výjimečné, z důležitých důvodů, muže soud povolit uzavřít manželství nezletilému, který není plně svéprávný a dovršil šestnácti let. Uzavření manželství za trvání této zákonné překážky vede k jeho \textit{relativní neplatnosti}.
    \item \textbf{Nedostatek svéprávnosti} - manželství nemůže uzavřít osoba, jejíž způsobilost byla v této oblasti omezena. Manželství uzavřené za trvání této zákonné překážky bude stiženo \textit{relativní neplatností}.
    \item \textbf{Existující manželství, registrované partnerství či jiný obdobný svazek} - manželství nemůže uzavřít osoba za trvání jiného manželství, registrovaného partnerství, či jiného obdobného zahraničního svazku (např. občanský pakt solidarity ve Francii), které tato osoba dříve uzavřela. Uzavření manželství za trvání této zákonné překážky vede k jeho \textit{absolutní neplatnosti}.
\end{itemize}

Zákon poté definuje i následující \textbf{relativní zákonné překážky manželství}, které vylučují, aby osoba uzavřela manželství s konkrétní osobou:

\begin{itemize}
    \item \textbf{Příbuzenství} - manželství nemůže být uzavřeno mezi předky a potomky, ani mezi sourozenci; totéž platí o osobách, jejichž příbuzenství vzniklo osvojením. Uzavření manželství za trvání této zákonné překážky vede k jeho \textit{absolutní neplatnosti}.
    \item \textbf{Poručenství a pěstounství} - manželství nemůže být uzavřeno mezi poručníkem a poručencem, mezi dítětem a osobou, do jejíž péče bylo dítě svěřeno, nebo pěstounem a svěřeným dítětem. Manželství uzavřené za trvání této překážky je stiženo \textit{relativní neplatností}.
\end{itemize}

\section{Věcná práva}

\textbf{Věc} je definováno jako vše, co je rozdílné od osoby a slouží potřebě lidí. Specificky - člověk, osoba a zvíře není věc. Věci dělíme na \textbf{hmotné} (ovladatelná část vnějšího světa, která má povahu samostatného předmětu) a \textbf{nehmotné} (práva, jejichž povaha to připouští, a jiné věci bez hmotné podstaty), a na \textbf{nemovité} (pozemky a podzemní stavby) a \textbf{movité} (ostatní). \textbf{Právo stavby} je věcným právem, které je nehmotnou nemovitou věcí a zahrnuje budovu postavenou na cizím pozemku, omezeným na 99 let.

V rámci věcného práva definujeme pojmy \textit{uživání} (využití věci k vybraným aktivitám, pro předání dál můžu využít \textit{nájemní smlouvu}) a \textit{požívání} (čerpání výsledků z daného pozemku, pro předání dál můžu využít \textit{pachtovní smlouvu}) věci. Lze definovat věcná práva k věci cizí pomocí \textbf{věcných břemen}. Ta se dělí na \textit{služebnosti} (oprávněni užívat cizí věc, lze zřídit buďto osobě nebo mezi vlastníky pozemku služebného a panovního) a \textit{reálná břemena} (povinnost vykonávat určité plnění).

\subsection{Vlastnické právo}

Právo vlastnictví nastaví úplné nebo částečné právní panství nad konkrétní věcí. Právo dodržuje pravidlo \textbf{erga omnes} - právo příslušní jen jedné konkrétní osobě, ale všichni ostatní (bez jejich konkrétního určení) jsou povinni toto právo respektovat.

Každý má právo vlastnit majetek. Vlastnictví zavazuje a nesmí být zneužito na újmu práv druhých anebo v rozporu se zákonem chráněnými obecnými zájmy. Výkon vlastnického práva nesmí poškozovat lidské zdraví, přírodu a životní prostředí.

Vlastnictví lze nabývat na základě smlouvy - \textit{převodem}, \textit{přechodem} (podle zákona nebo soudního rozhodnutí), \textit{přivlastněním} (pro věci ničí), \textit{vydržením} (nabytí po 3 letech movitých věcí / 10 letech nemovitých věcí při dobré víře, že věc vlastním), \textit{přírůstkem} nebo \textit{děděním}.

\textbf{Spoluvlastnictví} zahrnuje situaci, kdy jednu věc vlastní současně více osob. Platí, že z věcí nakládají jako jedna osoba, každý má právo k celé věci. Lze definovat spoluvlastnický podíl jako vyjádření míry účasti každého spoluvlastníka na vytváření společné vůle a právech/povinnostech vyplývajících ze spoluvlastnictví věci

\section{Závazková práva}

\textbf{Závazek} je právní vztah, kde \textit{dlužník} je povinen něco (tzv. \textit{dluh}) dát, něco konat, něčeho se zdržet nebo něco strpět a \textit{věřitel} je oprávněn to od něho požadovat a vymáhat své právo (\textit{pohledávka}). Závazky vznikají ze \textit{smluv} nebo \textit{protiprávního jednání}.

\textbf{Smlouva} je projev vůle stran zřídit mezi sebou závazek a řídit se obsahem smlouvy. Smlouva vzniká \textbf{kontraktační}m \textbf{proces}em, kde navrhovatel (\textit{oferent}) provede návrh na uzavření smlouvy (\textit{nabídka}) druhé straně (\textit{oblát}). Samotný úmysl uzavřít určitou smlouvu je závazný. Akceptace nabídky je omezena \textit{akceptační lhůtou} stanovenou buďto přesně, nebo přiměřeně povaze smlouvy.

V rámci doby trvání závazku lze provést změny v osobě věřitele (nevyžaduje souhlas dlužníka), změny v osobě dlužníka (vyžaduje souhlas věřitele s výjimkou \textit{přistoupení k dluhu}) a změna obsahu.

Závazek zaniká splněním (potřebně řádně a včas, možné domluvit \textit{fixní závazek} s neměnným termínem, potvrzené \textit{kvitanci} splnění dluhu). Závazek nezaniká vadný plněním (bez stanovených vlastností, neupozorněním na vady, uvedením v rozpor nebo zcizením), vadu je nutné bez zbytečného odkladu poté, kdy je dána možnost věc prohlédnout a vadu zjistit, nejpozději do šesti měsíců. \textit{Nápadná vada} zřejmá již při uzavírání smlouvy patří k tíži nabyvatele. Z jiných důvodů může závazek končit \textit{dohodou}, \textit{započtením}, \textit{odstupným}, \textit{uplatněním poukázky}, \textit{splynutím} (fúzí právnických osob), \textit{prominutím}, \textit{výpovědí}, \textit{odstoupením} nebo \textit{nemožností plnění}.

\textbf{Zajištění}m \textbf{dluhu} rozumíme závazek třetí osoby směrem k věřiteli nebo v jeho prospěch za splnění dluhu. \textbf{Utvrzení}m \textbf{dluhu} dlužník sjedná s věřitelem pro případ smluvní pokutu nebo uznání dluhu.

\textbf{Závazky z deliktů} vzniká porušením právní povinnosti a vznikem škody nebo újmy nemajetkové povahy. Škoda má být odčiněna uvedením do předešlého stavu; není-li to dobře možné, poté v penězích za škodu a ušlý zisk. Nemajetková újma je odčiněna přiměřeným zadostiučiněním.
