\epigraph{Vytvořeno podle přednášek \textbf{RNDr. Petra Tučníka, Ph.D.} v kombinovaném studiu v zimním semestru 2021/2022 a podle materiálů dostupných ke studiu v systému UHK Oliva.}{}

\section{Agenty a Multiagentní systémy}

\textbf{Agent} je program, který má autonomní chování, vnímání prostředí a adaptaci na změny - cokoliv, co vnímá své \textbf{prostředí} prostřednictvím \textbf{senzorů} a koná v tomto prostředí pomocí \textbf{aktuátorů}. Agenty vynikají autonomií, případně možností vnímat sociální kontext. Příklady agentů reprezentují \textit{robotický agent}, \textit{softwarový agent} a další.

\textbf{Vjem} je označení senzorických vstupů agenta v kterémkoli okamžiku. Agent je specifikován, jestliže pro libovolnou sekvencí vjemů je definována akce. Toto chování agenta je popsáno \textbf{funkcí agenta}, ta je realizována \textbf{programem agenta}.

\textbf{Racionální agent} je takový agent, který má v rámci své funkce přiřazenou \textit{správnou akci} pro každý věk. Správnost akce se hodnotí podle toho, zda zvyšuje úspěšnost agenta při plnění jeho úkolu - \uv{výkonnost agenta}. Výkonnost agenta se měří podílem správných stavů vnitřního fungování agenta.

\textbf{Úkolové prostředí} jsou problémy, pro které agenty jsou řešení. Úkolová prostředí jsou popsána \textbf{PEAS} popisem - výkonnost (performance), prostředí (environment), aktuátory (actuators) a senzory (sensors).

\begin{itemize}
\item \textbf{Pozorovatelnost} relevantních aspektů - \textit{plně pozorovatelné} a \textit{částěčně pozorovatelné}.
\item \textbf{Determinismus} - \textit{deterministické} a \textit{stochastické}.
\item \textbf{Dělení zkušeností} - \textit{epizodické} a \textit{sekvenční}.
\item \textbf{Stabilita prostředí} - \textit{statické} a \textit{dynamické}.
\item \textbf{Kontinuita} - \textit{diskrétní} a \textit{kontinuální}.
\item \textbf{Četnost agentů} - \textit{single agentní}, \textit{konkurenční multiagent} a \textit{kooperativní multiagentntní}.
\end{itemize}

\subsection{Typy agentů}

\subsubsection{Ryze reaktivní agenty}
Tyto agenty na základě aktuálního vjemu volí konkrétní akci.

$$ KDYŽ(vjem) \rightarrow PAK(akce) $$

Vhodné pro plně pozorovatelné systémy.

\subsubsection{Agenty s vnitřní reprezentací prostředí}
Tyto agenty uchovávají reprezentaci vnitřního stavu, závislou na minulých vjemech. Tato reprezentace vytváří \textbf{model světa}, na které agent reaguje.

\subsubsection{Agenty řízené cílem}
Tento agent kromě modelu světa má navíc \textbf{cíl akce}, který se agent svým chováním snaží dosáhnout. Pokud cíl není bezprostřední, musí agent vyhledat sekvenci akcí vedoucí k dosažení cíle. Tyto postupy se nazývají \textbf{prohledávání} a \textbf{plánování}.

\subsubsection{Agenty řízené užitečností}
Tyto agenty navíc řeší i kvalitu rozhodnutí, která vedou k cíli. Mají definovánu \textbf{odměňovací funkci}, která rozlišuje mezi horším a lepším řešením vzniklé situace. Tato funkce přiřazuje stavům reálné číslo, které reprezentuje kvalitu řešení.

\subsubsection{Učící se agenty}

\textbf{Učení} umožňuje agentu adaptovat se v neznámém prostředí a optimalizovat výkon. TODO Zpětná vazba a pojmy

\section{Architektury MAS}

Volba struktury má významný dopad na efektivitu jeho fungování. Základní myšlenkou je mí† systém implementačně co nejjednodušší, ale takový, aby stačil řešit určený problém.


\begin{itemize}
\item \textbf{Hierarchie} - Typicky stromová struktura bez laterálních vazeb. Vzniká latence z důvodu předávání informací. Zavedení architektury vede k nižšímu množství interakcí mezi agenty, možnost vytváření \textit{omezení} pro chování agentů pomocí definice \textit{rolí} a diverzifikace informací poskytované agentům.
    \begin{itemize}
        \item Vzor \textbf{Kontraktační síť} - úkol je rozložen na podúkoly, pro jejich řešení jsou uzavírány krátkodobé kontrakty s agenty nižších úrovní.
    \end{itemize}
\item \textbf{Holarchie} - Základním prvkem je \textit{holon} - částečně autonomní zapouzdřené jednotky skládající se z koordinačního agenta, který zakryje skupinu agentů, se kterými spolupracuje.
\item \textbf{Koalice} - Dočasné struktury zaměřené na specifický cíl s plochou strukturou (případně s koordinačním agentem).
\item \textbf{Týmy} - Skupina kooperující agentů s prioritizací užitku skupiny. Individuální akce mají být koordinovány tak, aby byly v souladu s cílem týmu. Arbitrární uspořádání.
    \begin{itemize}
        \item Vzor \textbf{Matchmaker} - TODO
    \end{itemize}
\item \textbf{Kongregace} - Dlouhotrvající, plochá struktura agentů s podobnými charakteristikami. Pomáhá redukovat limit interakcí a složitost vyhledávání agentů. Vytváření kongregací probíhá pomocí štítků.
\item \textbf{Societa} - Otevřené systémy definující struktura a řád, kde konkrétní interakce jsou flexibilní. Lokalizace pravidel MAS pro konkrétní prostředí / možnost definice násobných pravidel.
\item \textbf{Federace} - skupina agentů předá část své autonomie jinému agentovi, který skupinu reprezentuje. Členové skupiny komunikují pouze ze svým zástupcem.
    \begin{itemize}
        \item Vzor \textbf{Broker} - TODO
    \end{itemize}
\item \textbf{Trhy} - obsahují agenty kupující zdroje a prodejci, kteří zpracovávají poptávky a určují rozdělení zdrojů. Výhodné pro alokační úlohy a cenotvorbu.
\item \textbf{Maticové organizace} - kombinace manažerských a řešitelských agentů na násobné vazby mezi jednotlivými objekty. Cenou vyšší komplexity dochází k optimalizaci řešitelských zdroj.
\item \textbf{Složené organizace} - kombinace předchozích architektur.
\end{itemize}